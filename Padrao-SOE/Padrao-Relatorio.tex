\documentclass[a4paper,11pt,final]{article}
\pdfoutput=1 % if your are submitting a pdflatex (i.e. if you have
             % images in pdf, png or jpg format)

\usepackage{jheppub} % for details on the use of the package, please
                     % see the JHEP-author-manual

\usepackage[T1]{fontenc} % if needed
\usepackage[brazil]{babel} 
\usepackage[latin1,utf8]{inputenc}
\usepackage{graphicx} 
\usepackage{tikz}
\usetikzlibrary{arrows,automata}
\usepackage{algorithm}
\usepackage{algorithmic}
\usepackage{enumerate}
\usepackage{multicol}
\usepackage{multirow}


%\subheader{Review}
\title{\boldmath Relatório 1: Sincronização de Processos}


%% simple case: 2 authors, same institution
 \author{José da Silva,}
 \author{Maria da Silva}
 \affiliation{Universidade Federal do Paraná,\\Engenharia Elétrica com Ênfase em Sistemas Eletrônicos Embarcados}

% more complex case: 4 authors, 3 institutions, 2 footnotes
%%\author[a,b,1]{F. Irst,\note{Corresponding author.}}
%%\author[c]{S. Econd,}
%%\author[a,2]{T. Hird\note{Also at Some University.}}
%%\author[a,2]{and Fourth}

% The "\note" macro will give a warning: "Ignoring empty anchor..."
% you can safely ignore it.

%%\affiliation[a]{One University,\\some-street, Country}
%%\affiliation[b]{Another University,\\different-address, Country}
%%\affiliation[c]{A School for Advanced Studies,\\some-location, Country}
%%
% e-mail addresses: one for each author, in the same order as the authors
\emailAdd{jose@ufpr.br}
\emailAdd{maria@ufpr.br}


\abstract{Este padrão de relatório está baseado no padrão utilizado pelo \emph{Journal of High Energy Physics}, disponível em http://jhep.sissa.it/jhep/.
Este documento descreve o padrão a ser empregado para apresentação de relatórios.
No resumo, procure descrever o problema, o método usado para resolver e aponte as principais conclusões. 
Seja objetivo no resumo, não entre em detalhes, mas não esqueça de apresentar as questões chave tratadas.
Ofereça todos os elementos para permitir que o leitor decida se deve prosseguir lendo o trabalho em detalhes.
Este roteiro foi preparado utilizado \LaTeX.}
 


\begin{document} 
\maketitle
\flushbottom

\section{Introdução}\label{sec:intro}

A introdução deve incluir:

\begin{itemize}
 \item Descrição do problema sendo tratado;
 \item Objetivos a serem atingidos;
 \item Metodologia empregada (materiais e métodos).
\end{itemize}

Para citar referências cruzadas neste documento, faça como o exemplo: ver seção~\ref{sec:intro}.
Para citar referências bibliográficas: com uma referência~\cite{a} ou com múltiplas referências\cite{a,b,c,d}. Este formato é usado em muitas revistas e também é admitido pela ABNT.
Exemplo de equações, veja a Equação \eqref{eq:x}.

\begin{equation}
\label{eq:x}
\begin{align}
w_1=  \frac{q^4}{A}, w_2=  \frac{p \cdot q^3}{A}, w_3=  \frac{p^2 \cdot q^2}{A}, w_4=  \frac{p^3 \cdot q}{A}, w_5=  \frac{p^4}{A}\\
A=  q^4+p \cdot q^3 + p^2 \cdot q^2 + p^3 \cdot q + p^4\\
\end{align}
\end{equation}

Explique e cite as figuras e tabelas no texto. Exemplo: a figura~\ref{fig:qei} trata de ... a tabela~\ref{tab:i} mostra ...

\begin{figure}[ht] 
\centering
\scalebox{1.0}{
\begin{tikzpicture}[>=stealth',shorten >=1pt,auto,node distance=2cm]
  \node[state] 			(R)      			{R};
  \node[state]       	(B) [below left of=R]  	{B};
  \node[state]         	(P) [below right of=R] 	{P};


  \path[->] (R)  edge [bend right] node [swap] {\textit{solicita E/S}} (B)
                 edge [bend left]  node {\textit{escalonador}}   (P)
	        (B)  edge [bend right] node [swap] {\textit{fim E/S}}  (P)
	        (P)  edge [bend left] node {} (R);
\end{tikzpicture}}
\caption{Diagrama de estados possíveis para processos}\label{fig:qei}
\end{figure}



\begin{table}[ht]
\centering
\scalebox{0.8}{
\begin{tabular}{|c|c|c|c|c|c|c|c|c|c|c|c|}
\hline
\multirow{3}{*}{Vídeo} & \multirow{3}{*}{GOP} & \multicolumn{10}{c|}{Background UE}                                                                                             \\ \cline{3-12} 
                       &                      & \multicolumn{2}{c|}{10} & \multicolumn{2}{c|}{20} & \multicolumn{2}{c|}{30} & \multicolumn{2}{c|}{40} & \multicolumn{2}{c|}{60} \\ \cline{3-12} 
                       &                      & PSNR       & SSIM       & PSNR       & SSIM       & PSNR       & SSIM       & PSNR       & SSIM       & PSNR       & SSIM       \\ \hline
\multirow{4}{*}{Ice}   & (15,0)               &            &            &            &            &            &            &            &            &            &            \\ \cline{2-12} 
                       & (30,0)               &            &            &            &            &            &            &            &            &            &            \\ \cline{2-12} 
                       & (45,0)               &            &            &            &            &            &            &            &            &            &            \\ \cline{2-12} 
                       & (90,0)               &            &            &            &            &            &            &            &            &            &            \\ \hline
\end{tabular}}
\caption{Exemplo de tabela.}\label{tab:i}
\end{table}

Use algoritmos para explicar a solução adotada usando pseudo código. Explique os algoritmos e faça a ligação com o código fonte, que deve ser preferencialmente apresentado no apêndice.


\section{Seção}
Divida apropriadamente seu trabalho em seções e subseções.

\subsection{Subseção}
Entre seções e sub-seções sempre deve existir texto.

\subsubsection{Subsubseção}
Texto da sub-sub-seção.

\paragraph{Parágrafo} se for necessário dividir uma subsubseção, use um parágrafo não numerado para evitar itens numerados muito longos.


\appendix
\section{Apendice}
Não esqueça o título no apêndice. Use o apêndice para informações complementares.

\bibliographystyle{IEEEtran}
\bibliography{ref.bib}

\end{document}
